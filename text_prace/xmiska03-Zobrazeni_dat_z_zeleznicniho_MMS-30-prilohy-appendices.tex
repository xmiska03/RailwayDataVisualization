% Tento soubor nahraďte vlastním souborem s přílohami (nadpisy níže jsou pouze pro příklad)

% Pro kompilaci po částech (viz projekt.tex), nutno odkomentovat a upravit
%\documentclass[../projekt.tex]{subfiles}
%\begin{document}

% Umístění obsahu paměťového média do příloh je vhodné konzultovat s vedoucím
%\chapter{Obsah přiloženého paměťového média}

%\chapter{Manuál}

\chapter{Projektový súbor}
\label{ch:project_file}

Pre nahrávanie dát do implementovanej aplikácie bol navrhnutý jednoduchý jednoúrovňový projektový súbor vo formáte TOML, ktorého štruktúra je popísaná tabuľkou \ref{tab:projektovy_subor}.

\begin{landscape}
\begin{longtable}{>{}p{20em}|>{}p{8em}|>{}p{28em}}
    \caption{Štruktúra projektového súboru.}
    \label{tab:projektovy_subor} \\
    
    Názov premennej & Dátový typ & Popis \\ \hline \hline
    \texttt{project\_path} & reťazec & Absolútna cesta k~priečinku s~projektom. Všetky ostatné cesty sú relatívne vzhľadom k~tomuto priečinku. \\ \hline
    \texttt{postprocess\_pcd\_path} & reťazec & Relatívna cesta k~súboru s~mračnom bodov typu postprocess vo formáte \texttt{pcd}. \\ \hline
    \texttt{realtime\_pcd\_path} & reťazec & Relatívna cesta k~priečinku so súbormi s~mračnom bodov typu real-time vo formáte \texttt{pcd}. \\ \hline
    \texttt{realtime\_pcd\_filename\_prefix} & reťazec & Prefix názvov týchto súborov\footnote{Jednotlivé súbory musia byť pomenované \texttt{[prefix]\_0.pcd}, \texttt{[prefix]\_1.pcd}, atď.}. \\ \hline
    \texttt{realtime\_pcd\_files\_cnt} & prirodzené číslo & Počet týchto súborov. \\ \hline
    \texttt{realtime\_pcd\_timestamps\_path} & reťazec & Relatívna cesta k~súboru s~časovými razítkami mračna bodov typu real-time vo formáte \texttt{txt}\footnote{Každý riadok súboru musí byť vo formáte \texttt{[cislo\_riadku] [casove\_razitko]}.}. \\ \hline
    \texttt{video\_path} & reťazec & Relatívna cesta k~súboru s~videom vo formáte \texttt{mp4}, s~kódovaním H.264. \\ \hline
    \texttt{vector\_data\_path} & reťazec & Relatívna cesta k~priečinku so súbormi s~vektorovými dátami vo formáte \texttt{csv}. \\ \hline
    \texttt{translations\_path} & reťazec & Relatívna cesta k~súboru s~transláciami kamery vo formáte \texttt{csv}. \\ \hline
    \texttt{rotations\_path} & reťazec & Relatívna cesta k~súboru s~rotáciami kamery vo formáte \texttt{csv}. \\ \hline
    \texttt{timestamps\_path} & reťazec & Relatívna cesta k~súboru s~časovými razítkami polôh kamery vo formáte \texttt{csv}. \\ \hline
    \texttt{profile\_translations\_path} & reťazec & Relatívna cesta k~priečinku so súbormi s~transláciami prejazdného profilu vo formáte \texttt{csv}\footnote{Každý riadok súboru musí byť vo formáte \texttt{[x] [y] [z]}. Jednotlivé súbory musia byť pomenované \texttt{[prefix]\_25.csv}, \texttt{[prefix]\_50.csv}, \texttt{[prefix]\_75.csv}, \texttt{[prefix]\_100.csv}.}. \\ \hline
    \texttt{profile\_translations\_filename\_prefix} & reťazec & Prefix názvov týchto súborov. \\ \hline
    \texttt{profile\_rotations\_path} & reťazec & Relatívna cesta k~priečinku so súbormi s~rotáciami prejazdného profilu vo formáte \texttt{csv}\footnote{Každý riadok súboru musí byť vo formáte \texttt{[yaw] [pitch] [roll]}.}. \\ \hline
    \texttt{profile\_rotations\_filename\_prefix} & reťazec & Prefix názvov týchto súborov. \\ \hline
    \texttt{calibration\_matrix} & pole 9 čísel & Kalibračná matica. \\ \hline
    \texttt{far\_plane} & prirodzené číslo & Vzdialenosť zadnej orezávacej roviny v metroch. \\ \hline
    \texttt{distortion\_parameters} & pole 5 čísel & Parametre skreslenia kamery.
\end{longtable}
\end{landscape}

%\chapter{Plakát}

% Pro kompilaci po částech (viz projekt.tex) nutno odkomentovat
%\end{document}
