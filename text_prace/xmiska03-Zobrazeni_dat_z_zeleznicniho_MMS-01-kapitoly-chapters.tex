% Pro kompilaci po částech (viz projekt.tex), nutno odkomentovat a upravit
%\documentclass[../projekt.tex]{subfiles}
%\begin{document}

\chapter{Úvod}

\chapter{Vizualizácia dát z mobilných mapovacích systémov}

[popis, aké dáta produkujú m. m. s.]

\section{Dierkový model kamery -- teoretický základ zobrazenia}

[...]

\section{Existujúce možnosti zobrazenia dát z mobilných mapovacích systémov}

[popis základných knižníc aj programov, ako napr. CloudCompare]

\subsection{Riešenia dostupné v jazyku Python}
\subsubsection{Framework deck.gl a jeho nadstavba Pydeck}

Javascriptový framework deck.gl je určený na zobrazovanie dát najmä na mapových podkladoch. Vyznačuje sa vysokou presnosťou a výkonnosťou. Pre akceleráciu využíva rozhrania WebGPU a WebGL2 \cite{deck.gl_documentation}.

Vizualizácia dát v deck.gl sa skladá z dvoch základných častí:
\begin{itemize}
    \item Vrstvy (\texttt{Layers}). Do vrstiev sa ukladajú zobrazované dáta. Framework deck.gl ponúka vyše 30 preddefinovaných typov vrstiev. Pre túto prácu je významná najmä vrstva \texttt{PointCloudLayer}, ktorá je určená na zobrazenie mračna bodov.
    \item Pohľad (\texttt{View}). Definuje vlastnosti kamery, napríklad zorné pole a prednú a zadnú orezávaciu rovinu (\emph{near plane} a \emph{far plane}).
    Časť \texttt{ViewState} určuje polohu a smer pohľadu kamery. Typ pohľadu definuje spôsob interakcie vizualizácie s používateľom, napríklad pre zobrazenie trate z pohľadu strojvedúceho je ideálny typ \texttt{FirstPersonView}.
\end{itemize}

Napriek tomu, že je framework deck.gl primárne určený pre použitie v Javascripte, je možné ho použiť aj v jazyku Python, a to pomocou knižnice \textbf{Pydeck}. Tá je pomerne jednoduchá a podstatou jej činnosti je, že prevedie kód napísaný v jazyku Python do formátu JSON. Framework deck.gl má totiž modul @deck.gl/json, ktorý prijíma reprezentáciu vizualizácie vo formáte JSON a transformuje ju do javascriptového kódu (na definície funkcií a deck.gl objektov)\footnote{Ukážka rozhrania modulu @deck.gl/json je na \url{https://deck.gl/playground}.}.

Knižnica Pydeck je dobrým prostriedkom na vytvorenie jednoduchých vizualizácií, s ktorými môže používateľ interagovať pohybmi myši. Jej možnosti sú však oproti pôvodnému frameworku deck.gl veľmi obmedzené. Nie je však vhodná na vytváranie zložitejších animácií s veľkým množstvom dát, pretože sa aj po tej najmenšej zmene musia dáta a definícia vizualizácie nanovo prevádzať do formátu JSON a následne na javascriptový kód, čo je veľmi časovo náročné.

\section{Frameworky pre tvorbu interaktívnej webovej aplikácie v jazyku Python}

\subsubsection{Streamlit a Dash}

Streamlit a Dash sú frameworky, ktoré majú rovnaké zameranie: oba slúžia na tvorbu webových aplikácií pre prácu s dátami (\emph{data apps}) v jazyku Python. Dash je oproti Streamlitu na nižšej úrovni abstrakcie, pretože sám o sebe nemá žiaden vizuálny štýl a mnohé jeho komponenty sa priamo mapujú na HTML elementy, napríklad \texttt{dash.html.Div} a \texttt{dash.html.H1} \cite{streamlit_documentation}\cite{dash_documentation}.

Oba frameworky majú podporu pre Pydeck, u Streamlitu je priamo k dispozícii element \texttt{st.pydeck\_chart} a k Dashu je možné pridať knižnicu \textbf{Dash Deck}. Ukázalo sa však, že \texttt{st.pydeck\_chart} podporuje iba pohľad \texttt{MapView}, ktorý je určený na zobrazenie dát na mape a nedá sa použiť na perspektívne zobrazenie bodov v trojrozmernom priestore.

Dash Deck má navyše tú výhodu, že umožňuje vynechať Pydeck a definovať zobrazenie pomocou slovníkov so štruktúrou zodpovedajúcou tej, ktorú vyžaduje modul @deck.gl/json. To trochu zefektívni vykonávanie zmien vo vizualizácii, keďže to umožní upravovať jednotlivé časti zvlášť.

[obrázok ilustrujúci transformácie]

\chapter{Návrh aplikácie pre vizualizáciu dát z železničného mobilného mapovacieho systému}

\section{Návrh používateľského rozhrania}

\chapter{Implementácia navrhnutej aplikácie}

\section{Experimenty a vyhodnotenie výsledkov}

\chapter{Záver}


%===============================================================================

% Pro kompilaci po částech (viz projekt.tex) nutno odkomentovat
%\end{document}

